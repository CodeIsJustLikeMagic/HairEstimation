\documentclass[german,a4paper,12pt]{llncs}
\setcounter{tocdepth}{2}
\makeatletter
\renewcommand*\l@author[2]{}
\renewcommand*\l@author[2]{}
\makeatletter
\usepackage[utf8]{inputenc}
\usepackage[backend=biber,sorting =none]{biblatex}
\usepackage{csquotes}
\usepackage{graphicx}
\usepackage{babel}
\usepackage{subcaption}
\usepackage{mwe}

\usepackage{parskip}
\usepackage{float}

%\usepackage{hyperref}
\usepackage{filecontents}


\begin{filecontents}{references.bib}
	
@article{visualScale,
	author = {Martínez-Velasco, María and Vázquez-Herrera, Norma and Maddy, Austin and Asz-Sigall, Daniel and Tosti, Antonella},
	year = {2017},
	month = {02},
	pages = {},
	title = {The Hair Shedding Visual Scale: A Quick Tool to Assess Hair Loss in Women},
	volume = {7},
	journal = {Dermatology and Therapy},
	doi = {10.1007/s13555-017-0171-8}
}

@inproceedings{GreenScreenHair,
	author = {Legendre, Chloe and Krissman, David and Debevec, Paul},
	year = {2017},
	month = {07},
	pages = {1-2},
	title = {Improved chromakey of hair strands via orientation filter convolution},
	isbn = {978-1-4503-5015-0},
	doi = {10.1145/3102163.3102200}
}
@article{seasoalShedding,
	
	author = {Hsiang, E.Y. and Semenov, Yevgeniy and Aguh, Crystal and Kwatra, S.G.},
	year = {2017},
	month = {10},
	pages = {},
	title = {Seasonality of hair loss: a time series analysis of Google Trends data 2004 to 2016},
	volume = {178},
	journal = {British Journal of Dermatology},
	doi = {10.1111/bjd.16075}
}

@inbook{chemicalAlopecia,
	author = {Maibach, Howard and Yamaguchi, Ian},
	year = {2012},
	month = {01},
	pages = {1935-1942},
	title = {Chemically Induced Hair Loss/Alopecia},
	isbn = {978-3-642-02034-6},
	journal = {Kanerva's Occupational Dermatology, Second Edition},
	doi = {10.1007/978-3-642-02035-3_205}
}

@article{ironDeficiency,
	author = {Trost, Leonid and Bergfeld, Wilma and Calogeras, Ellen},
	year = {2006},
	month = {06},
	pages = {824-44},
	title = {The diagnosis and treatment of iron deficiency and its potential relationship to hair loss},
	volume = {54},
	journal = {Journal of the American Academy of Dermatology},
	doi = {10.1016/j.jaad.2005.11.1104}
}

https://matplotlib.org/
https://numpy.org/

\end{filecontents}

\addbibresource{references.bib}
\title{Exposee}
\subtitle{Automatisches Schätzen der Haaranzahl in ausgefallenen Haarbüscheln}
\author{\parbox{.9\textwidth}{\centering 
		\large Janelle Pfeifer \\
		\small Delpstraße 28\\
		53359 Rheinbach \\
		janelle.pfeifer@smail.inf.h-brs.de}}
\institute{\parbox{.9\textwidth}{\centering 
		\large Hochschule Bonn-Rhein-Sieg \\
		\normalsize Institute of Visual Computing \\ 
		\small Fachbereich Informatik \\
		Studiengang: Informatik (B.SC.)\\
		\normalsize Rheinbach, 15.02.2020}}
\date{Rheinbach, 8.1.2020}
\begin{document}
	%{\let\newpage\relax\maketitle}
	\maketitle
	\newpage
	\tableofcontents
	\newpage

\begin{abstract}
	am ende hier ein abstract hin
\end{abstract}

\section{Einleitung}
Die Einleitung ist mit der wichtigste Teil der Arbeit, da sie den Leser motivieren soll die vorliegende Arbeit weiter z lesen. Sie sollte neben einer Motivation bzw Problembeschreibung auch das Ziel der Arbeit beschreiben.Zusätzlich sollten die Fabgebiete die die arbeit betreffen und deren Bedeutung genannt werden. Wichtig ist eine klare Formulierung der Zielstellung und des Lösungsansatzes so wie abschließend eine Beschreibung der Gliederung der Arbeit.

Das Ausfallen von Haarsträhnen ist ein natürlicher Teil des Haarwachstum Zyklus. Durchschnittlich verliert 50 bis 100 Haarsträhnen am Tag. Vermehrter Haarverlust kann ein Indikator von Krankheit oder anderen Problemen sein. Da ist es von Vorteil das früher zu erfahren.

Der Haarverlust wird von unterschielidchen faktoren beeinflusst. So gibt es seasonale unterschiede. Haartextur und länge verändert den eindruck den eine person. so ist es nicht leicht den normalen haarverlust für eine person festzustellen 

computer vision. was ist das?

das paper gibt einstig in haar estimation mit computer vision. In dem Paper \blockquote{The Hair shedding visual scale: A quick tool to assess hair loss in Women} wird eine Methode beschrieben, in der Frauen anhand von Bildern den Umfang ihres täglichen Haarausfalls bestimmen können. Dabei werden den Frauen Bilder von abgezählten, ausgefallenen Haarbüscheln gezeigt, die ihrer eigenen Haarlänge entsprechen. Die Frauen wählen das Foto aus, welches ihrem persönlichen täglichen Haarausfall entspricht. Es wurde eine Korrelation festgestellt zwischen Frauen, die klinisch bestätigt Haarverlust erfahren und den Bildern, die sie ausgewählt haben.\cite{visualScale}

In dieser Arbeit wird ein verfahren vorgestellt mit dem eine langzeitüberwachung des täglichen Haarverlustes duchgeführt werden kann.

\section{Grundlagen}

\subsection{Haarausfall}
Wie in dem Paper \blockquote{The Hair shedding visual scale: A quick tool to assess hair loss in Women} beschrieben, kann man den täglichen Haarausfall als Indikation des gesamten Haarausfalls nehmen.\cite{visualScale}

Haarausfall wird von mehrere Faktoren bestimmt und ist grundlegend gesund. Haarausfall ist nicht uniform. Er kann sich abhängig von dem Haartyp, der Haarlänge und der Haarpflege-Methode jeden Tag ändern und Muster aufweisen. 

Eine Person die ihre Haare einmal in der Woche kämmt kann zum Zeitpunkt des Kämmens mehr Haarausfall erwarten, als eine Person die ihre Haare jeden Tag kämmt. 

Hinzukommen saisonale Unterschiede in dem Haarverlust.\cite{seasoalShedding}

Jeder Mensch hat ein eigenes Haarausfall-Muster. Abweichungen von diesem Muster können Anzeichen von Krankheiten und Mangelerscheinungen sein. Beispielsweise ist erhöhter Haarausfall ein Symptom von Eisenmangel.\cite{ironDeficiency}

\subsection{Computervision}
In dem Bereich der Bildverarbeitung gibt es viele Algorithmen, die es ermöglichen Objekte in einem Bild erkennbar zu machen und zu verarbeiten.

\subsubsection{Region Growth}
ist ein Verfahren, bei dem ein Bild in Regionen unterteilt wird. So werden zum Beispiel Sektionen mit ähnlicher Farbe herausgestellt. 

\subsubsection{Skelettierung}
reduziert Objekte in einem Bild auf eine Pixelbreite. So können Linien auf eine Pixelbreite geschrumpft werden.

\subsubsection{Dynamischer Schwellwert}
kann genutzt werden, um Schatten aus einem Bild zu entfernen. Ein dynamischer Schwellwert ist geeignet für kompliziertere Bildern und Belichtungssituationen.


\subsubsection{Hochpassfilter}
erhält hochfrequente Bildanteile, während Niederfrequente verschwindet bzw. abgeschwächt werden. 
Hochfrequent sind Bereiche mit schnellen Farb- und Helligkeitsveränderungen. Das tritt beispielsweise bei Linien und Kanten auf.

\subsubsection{Kantendetektion}

nutzt Hochpassfilter, um Kanten in einem Bild zu erkennen. 

\subsection{Erkennung von Haaren in Green-Screen Bildern}

In dem Paper\blockquote{Improved Chromakey of Hair Strands via Orientation Filter Convolution} wird eine Methode beschrieben wie Haarsträhnen von einem Green-Screen Hintergrund getrennt werden können. Dazu wird die Kontinuität der Haarsträhnen ausgenutzt. \cite{GreenScreenHair}

\section{Methode}

\section{Implementation}

mit python
nutzt plugin open cv und plugins ....

\section{Design und Implementation}

Das Ziel ist es die Einschätzung eines Haarbüschels, mithilfe von Computervision, zu automatisieren. Zunächst wird ein Kalibriervorgang durchgeführt, in dem der Nutzer abgezählte Haarbpüschel-Bilder eingibt. Dann kann die Menge der Haare in nicht abgezählten Haarbüscheln automatisch geschätzt werden.

Die Schätzungen werden zugehörig zum Datum gespeichert. Um einen Zeitlichen verlauf festzuhalten.
Zusätzlich zu den Schätzungen kann für jedes Datum eine Marker angegeben werden. So können regelmäßige Wiederholungen der Haarpflege, wie zum Beispiel eine Haarwäsche, notiert werden. Desweiteren können gesundheitliche Veränderungen oder Einnahme von Medikamenten vermerkt werden.

Das Programm wurde in Python implementiert und nutzt unter anderem das Plugin Open-cv.
Der Nutzer steuert das Programm über die Kommandozeile. 

Da die Haare jeder Person unterschiedlich sind, muss für jeden Nutzer eine eigene Kalibrierung durchgeführt werden. Um Vermischung von Daten unterschiedlicher Personen und häufige Neukalibrierungen zu vermeiden, kann das Programm beliebig viele Nutzer anlegen und verwalten. 

\section{Verarbeitung der Haar-Bilder}

Sowohl für die Kalibrierung als auch für die schätzung von Haar-Mengen müssen die eingegebenen Bilder zunächst verarbeitet werden und Daten, mit dennen auf die Haar-Menge geschlossen werden kann herausgestellt werden. Dazu werden Methoden der Bildverarbeitung verwendet.

Die Methode detect implementiert die Verarbeitung eines Bildes.
Als Eingabe wird von einem Haarbüschel auf einem einfarbige Hintergrund ausgegangen. Dabei sollte ein möglichst guter Kontrast zu der Haarfarbe gewählt worden sein. Beispielsweise ein weißer Hintergrund für dunkele haare oder ein schwarzer Hintergrund für blondes oder hellbraunes Haar. Die Ecken des Hintergrundes sind durch ein Symbol markiert. Vergleich Abbildung \ref{img:input}.

\begin{figure}[H]
	\centering
	\includegraphics[width=0.8\textwidth]{fig64/00IMG_20200406_153354_12_g_15.jpg}
	\caption[]{Input}
	\label{img:input}
\end{figure}

Im ersten Schritt der Bildverarbeitung wird das Bild auf die Markierungen in den Ecken zugeschnitten. Das ist in der Methode cropDots implementiert.
Über einen Aufruf der Open cv methode matchTemplate wird nach vorkommen der Markierung gesucht. Das Resultat ist ein Karte an Wahrscheinlichkeiten, die für jedes Pixel angibt, zu welcher Wahrscheinlichkeit sich dort die Markierung befindet. Die 4 Stellen an denen die Wahrscheinlichkeit am höchsten ist, werden als Eckpunkte angesehen. Siehe Abbildung \ref{img:cropDots}.
\begin{figure*}
	\centering
	\begin{subfigure}[b]{0.475\textwidth}
		\centering
		\includegraphics[width=0.2\textwidth]{fig64/TemplateDot.jpg}
		\caption[]{Das Aussehen der Markierung ist bekannt und wird über so ein Bild angegeben.}
		\label{img:TemplateDot} 
	\end{subfigure}
	\hfill
	\begin{subfigure}[b]{0.475\textwidth} 
		\centering
		\includegraphics[width=\textwidth]{fig64/01match res.png}
		\caption[]{Resultat von matchTemplate}
		\label{img:matchTempate}
	\end{subfigure}
	\vskip\baselineskip
	\begin{subfigure}[b]{0.475\textwidth}   
		\centering
		\includegraphics[width=\textwidth]{fig64/03rectImg.png}
		\caption[]{Die 4 erkannten Markierungen im Bild}
		\label{img:foundDots}
	\end{subfigure}
	\quad
	\begin{subfigure}[b]{0.475\textwidth}   
		\centering
		\includegraphics[width=0.9\textwidth]{fig64/03crop image.png}
		\caption[]{Resultat des Zuschneidens}
		\label{img:Crop}
	\end{subfigure}
	\caption[ der Prozess des Zuschneidens auf Markierungen in den Ecken ]
	{\small Der Prozess des Zuschneidens auf Markierungen in den Ecken} 
	\label{img:cropDots}
\end{figure*}


Durch das Zuschneiden der Bilder wird sichergestellt, das die Haare immer etwa einen gleichgroßen Hintergrund haben. So werden die Daten, die aus den Bildern gezogen werden, nicht verfälscht dadurch, das die Haare eventuell von unterschiedlichen Abständen fotografiert wurden. 

Nach dem Zuschneiden werden die Haare von dem Hintergrund getrennt.
Das ist in der Methode edgeProcess implementiert. 
Mithilfe von Canny Edge detection werden die Haare grob erkannt.
Weiße Pixel gelten als Haare, schwarze als Hintergrund. Siehe Abbildung \ref{img:Edges}.

\begin{figure}
	\centering
	\includegraphics[width=0.9\textwidth]{fig64/04edges.png}
	\caption[]{Canny Kantendetektion}
	\label{img:Edges}
\end{figure}

In der Methode hairPixelIntensity wird die Detektion verfeinert.
Auf das Resultat der Katendetektion wird eine Dilatation ausgeführt. Dadurch gelten mehr Pixel als Haare und eventuelle Löcher wachsen zusammen.

Da Haare halb-transparent sind, wird ihre Farbe intensiver, wenn sie sich überlagern. Sie weisen eine höhere Intensität auf.
Um die Haar-Intensität des Bildes zu extrahieren, wird ein Bild gebaut, in dem Hintergrund Pixel weiß sind, und Haarpixel ihren Graustufenwert aus dem Original Bild erhalten. Siehe Abbildung \ref{q:haironwhite}.

Wenn die Haare heller sind als der Hintergrund, es sich beispielsweise um blondes Haar auf schwarzem Hintergrund handelt haben die Haare eine Höhere intensität, wenn sich heller sind. Bei dunkelem Haar auf hellem Hintergund ist es gegenteilig. Die Intensität ist höher, wenn das Haar dunkler ist.
Damit beide Konstelationen zu einem ähnlichen Ergebnis kommen, werden die graustufenwerte invertiert, wenn die Haare hell sind. So ist auch bei ursprünglich hellem Haar, die Intensität höher, wenn die Graustufen dunkler sind.

\begin{figure}
	\centering
	\includegraphics[width=1.1\textwidth]{01quellc.PNG}
	\caption[]{Implementierung haare weiß, hintergrund schwarz}
	\label{q:haironwhite}
\end{figure}

Die Inversion dieses Bildes ergibt ein Bild mit schwarzem Hintergrund und Haaren, die Heller sind an stellen wo sie sich Überlagern. Siehe Abbildung \ref{img:Intensity}
\begin{figure}
	\centering
	\includegraphics[width=0.9\textwidth]{fig64/05intenstiy.png}
	\caption[]{Intensität-Abbildung der Haare}
	\label{img:Intensity}
\end{figure}

Es werden kleine vereinzelte Gebiete des Hintergrundes in der Kantendetection erkannt und gelten somit als Haare.
Um diese Gebiete zu erkennen und zu entfernen, wird der Watershed Algorithmus angewendet.   
Dabei werden Marker in alle verbundenen Gebiete gesetzt, die nicht schwarz sind. Der Watershed Algorithmus lässt die Regionen wachsen. 
Dann werden die Regionen entfernt, die im Verglich auf die Größe des Bildes sehr klein sind. Siehe Abbildung \ref{img:smallRegions}

Die Summe aller Graustufenwerte werden als Intensität gespeichert. Dabei wird der Hintergrund außer acht gelassen, weil es schwarz ist, und somit den Graustufenwert 0 hat.

\begin{figure}
	\centering
	\includegraphics[width=0.9\textwidth]{fig64/06small regions removed.png}
	\caption[]{Kleine Regionen Entfernt}
	\label{img:smallRegions}
\end{figure}

In der Methode hairPixelPercentage werden die Pixel gezählt die als Haare gelten. Zusätzlich wird der Prozentsatz an Haar Pixeln und die Intensität des Bildes im Vergleich zur Bildgröße gespeichert. 

Dann wird die Dichte der Haare Untersucht. In der Methode 
Dafür werden die Hintergrund Regionen bestimmt, die durch Haare getrennt werden. Dafür wird der Watershed Algorithmus angewendet.
Die Marker werden in alle zusammenhängenden Hintergrundgebiete gesetzt.
Dabei wird die Größe der Region, die die Haare umschließt gespeichert. Abbildung \ref{img:outerSection}.
Zusätzlich wird die Anzahl der Regionen die von Haaren umschlossen wird und deren Durchschnittliche Größe gespeichert.

\begin{figure}
	\centering
	\includegraphics[width=0.9\textwidth]{fig64/08outer section.png}
	\caption[]{Äußerte Region Isoliert und in weiß dargestellt.}
	\label{img:outerSection}
\end{figure}  

Als nächstes werden die Gebiete des Bildes untersucht, die eine erhöhte Haar-Dichte aufweisen.
Dafür wird das Intesität-Bild skelettiert. Siehe Abbildung \ref{img:skel}.


\begin{figure}
	\centering
	\includegraphics[width=0.9\textwidth]{fig64/09input intensity.png}
	\caption[]{Skelett des Intensität-Bildes}
	\label{img:skel}
\end{figure}

Die Hintergrund Regionen werden wie zuvor untersucht.

\begin{figure}
	\centering
	\includegraphics[width=0.9\textwidth]{fig64/09outer section.png}
	\caption[]{Äußerte Region des skelettierten Bildes}
	\label{img:skel}
\end{figure}

In der Methode denseAndLoosePerc wird die Haar-Intensität in den dichteren Regionen und in den weniger dichten Regionen berechnet und abgespeichert.

Für das Bild werden 28 Datenpunkte extrahiert und abgespeichert: 

\begin{itemize}
	\item intensitySum - Summe der Graustufenwerte des Intensität-Bildes
	\item intensityShare - Intensität pro HaarPixel
	\item hairpixels - Anzahl der Pixel, die Haare darstellen
	\item imagepixels - Gesammtanzahl der Pixel im Bild
	\item hairpixel percentage - Prozentsatz der Haarpixel im Bild
	\item number of sections - Anzahl an Hintergrund Regionen, die Watershed findet.
	\item number of background sections - Anzahl an Regionen, die den Hintergrund darstelle. (2 weniger als number of section. Watershed zählt Haare und übergänge zwischen Regionen mit)
	\item outerSectionSum - Summe an Pixeln, die sich in der Region befinden, die die Haare umschließt
	\item outerSectionPercentage - outerSectionSum/imagepixels
	\item innerSectionSum - Summa an Pixeln, die sich in Regionen befinden, die von Haaren umschlossen werden
	\item innserSectionAvgSize - Druchschnittliche Pixelanzahl, die sich in den umschlossenen Regionen befinden
	\item innerSectionAvgSize Percentage - innersectionAvgSize/imagepixels
	\item innerSectionSizeVariance - statistische Varianz der Pixelmengen, die sich in umschlossenen Regionen befinden
	\item innerSection standard deviation - Standardabweichung der Pixelmengen, die sich in umschlossenen Regionen befinden
	\item number of sections only concidering dense hair - Anzahl der Regionen, die Watershed findet nachdem der input Skelettiert wurde. Regionen werden von dichteren Haaren unterteilt.
	\item number of background sections inclosed by dense hair - Anzahl an Regionen, die den Hintergrund zwischen dichten Haaren darstellen
	\item dense hair outerSectionSum - Summe an Pixeln, die die Regionen umschließen, die von dichten Haaren umschlossen werden
	\item dense outerSectionPercentage - dense hair outerSectionSum / imagepixels
	\item dense innerSectionSum - Summe an Pixeln, die sich in den Regionen befinden, die von dichten Haaren umschlossen werden
	\item dense hair innserSectionAvgSize - Durschnittliche anzahl der Pixel, die sich in Regionen befinden, die von dichten Haaren umschlossen werden
	\item dense hair innerSectionAvgSize Percentage - dense hair innerSectionAvgSize / imagepixels
	\item dense innerSectionSizeVariance - statistische Varianz der Pixelmengen
	\item dense innerSection Size standard deviation - Standardabweichung der Pixelmengen
	\item denseHairSum - Summe an Pixeln, die als Dichte Haare angesehen werden
	\item looseHairSum - Summe an Pixeln, die als Haare angesehen werden, jedoch nicht als Dicht gelten
	\item IntensitySum in Loose Section - Summe der Graustufenwerte in nicht-dichten Regionen
	\item IntensitySum in Dense Section - Summe der Graustufenwerte in dichten Regionen
\end{itemize}

\subsection{Kalibrieren}

Die Kalibrierung ist in der Methode calibration impementiert.
Für die Kalibrierung werden mehrere Bilder als Input angegeben. Für jedes der Bilder wird die Menge der Haare in dem Titel des Bildes angegeben. 
Die Bilder werden alle mit detect Untersucht. Die Daten aus den Untersuchungen, sowie die Haar-Mengen werden abgespeichert.
ganzen Ordner kalibrieren. einzelne Bilder hinzufügen.

\subsection{Relation Daten zu Haar Menge}



\subsection{Schätzen eines Haar-Bildes}

Nachdem eine Kalibrierung durchgeführt wurde. Kann die Haar-Menge auf beliebigen Bildern geschätzt werden.
Das Schätzen ist in der Methode readFilesAndGuess implementiert.

Zuerst wird das zu schätzende Bild mithilfe von detect Untersucht.
Dann werden die Daten aus der Kalibrierung geladen. 
Die Daten werden teilweise miteinander verrechnet und dann auf die dazugehörigen Haar-Mengen abgebildet.
So entstehen eine Menge an Graphen, bei denen die Haar-Menge jeweils auf der y Achse liegt. Siehe Abbildung \ref{fig:mapping}. 

\begin{figure}
	\centering
	\includegraphics[width=1.2\textwidth]{fig64/gh4.PNG}
	\caption[]{Haar-Menge in Relation zur Intensität in dichten Regionen normalisiert durch gesamte Intensität und Region-Größe}
	\label{fig:mapping}
\end{figure}

Die Punkte in den Graphen werden durch Funktionen Approximiert. Die Funktionen können dann genutzt werden um Haar-Mengen zu schätzen.
Für jeden Graph werden 3 Funktionen berechnet. Eine Lineare Regression, eine Spline-Interpolation, und eine benutzerdefinierte Funktion.
Der Benutzer kann aus linear, logarithmisch und exponentiell aussuchen.
Siehe Abbildung \ref{fig:func}.

\begin{figure}
	\centering
	\includegraphics[width=1.2\textwidth]{fig64/g11_denseIntensitynorm.png}
	\caption[]{Funktionen, die Datenpunkte Approximieren. Spline-Interpolation, lineare Regression und exponentielle Funktion.}
	\label{fig:func}
\end{figure} 

Von den 28 Datenpunkten die aus den Bildern ausgewertet werden, werden 17 Genutzt um 12 Modelle zu erstellt. Diese Modelle wurden als aussagekräftig gefunden, um die Haarmenge schätzen zu können. Die folgenden x Werte werden verwendet:

\begin{itemize}
	\item hairpercent 
	\begin{itemize}
		\item Prozentsatz der Haarpixel im Bild
	\end{itemize}
	\item (hairpixels / hairSectionSize) * (1 - outerSectionPerc)
	HaarPixel dichte in der Region, in der sich Haare Befinden. Gibt dasselbe Ergebnis wie hairpercent.
	\item (hairpixels/hairSectionSize)*(1-outerSectionPerc)*hairPerc
	Dichte 
	
	\item (((denseHairSum / denseInnerSectionSize) * (1 - denseSectionperc))) = denseDensity
	\item denseDensity / looseDensity
	\item backgroundSectionNum*(1-outerSectionPerc)
	\item denseSectionNum*(1-outerSectionPerc)
	\item intensitySum*hairpixels*(1-outerSectionPerc)
	\item denseSectionAVGSize/(1-denseSectionperc)
	\item (denseSectionAVGSize/looseSectionAVGSize)/(1-outerSectionPerc)
	\item (denseIntensity/intensitySum)*(1-denseSectionperc)
	\item (denseIntensity/denseHairSum)*(1-denseSectionperc)
\end{itemize}

Siehe den Anhang für die graphische Darstellung der Modelle.

Für jedes der Modelle werden 3 Funktionen approximiert. Mit den Funktionen werden jeweils 3 Schätzungen für das zu s
chätzdene Bild berechnet. 

Ausreißer werden entfernt. Dazu werden zunächst alle Schätzungen von weniger als 0 Haaren entfernt. Zusätzlich werden statistische Ausreißer mithilfe eines Algorithmus basierend auf dem tschebyschef theorem entfernt. 

Der Mittelwert der übrig geblibenen Schätzungen ist das Endergebnis. 
Siehe Abbildung \ref{img:guess} für den Verlauf einer Schätzung auf der Kommandozeile.

\begin{figure}
  \centering
  \includegraphics[width=1.2\textwidth]{fig64/commandlineGuessProcess.PNG}
  \caption[]{Kommandozeilenaufruf und Verlauf einer Schätzung}
  \label{img:guess}
\end{figure} 

\subsection{Tests}

Primär wird mit meinen eigenen Haaren getestet. Weiter Haare werden von Freunden und Familie bereitgestellt.

Getestet wird an:
\begin{itemize}
	\item Knielangen, dunkelroten Haaren
	\item Hüftlangen, feinen, blonden Haaren
	\item Schulterlangen, brauen Haaren
	\item Den Bildern aus dem Paper \blockquote{The Hair shedding visual scale: A quick tool to assess hair loss in Women} 
\end{itemize}


\subsection{Genauigkeit/Fehler}


\section{Fazit}
Es wird eine praktische Methode gegeben eine Langzeitüberwachung von Haarausfall anzustellen. 

So wird der Anwender sich bewusst, welcher tägliche Haarausfall normal ist und welche Menge besorgniserregend ist. Haarausfall ist sehr variabel und wird durch viele Faktoren beeinflusst. 

Eine Langzeitüberwachung beruhigt den Nutzer bei saisonalem Haarausfall und gibt Hinweise auf die allgemeine Gesundheit der Haare. 

Die Langzeitüberwachung macht den Haarausfall transparenter und berechenbarer. So wird der Nutzer früher aufmerksam auf anomalen Haarausfall und kann darauf besser reagieren. Haarverlust ist ein Indikator für viele Krankheiten und kann somit als Warnsystem dienen.

%\ref{cm}

%\begin{figure}
%	\centering
%	\includegraphics[width=0.7\textwidth]{PraiseMe_Prototyp4.jpg}
%	%\includegraphics[width=0.7\textwidth]{bild.jpg}
%	\caption[]{Cover des Spiels}
%	\label{img:cover}
%\end{figure}

%\section*{Gruppenmitglieder}

%\begin{itemize}
%	\item Umlauf, Niklas    -   3D-Artist
%\end{itemize}

%\nocite{reaper}
\newpage
\printbibliography

\end{document}